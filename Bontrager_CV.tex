\documentclass[letterpaper,11pt,oneside]{article}
\usepackage[utf8]{inputenc}
\usepackage{setspace}
\usepackage{hyperref}

\usepackage{graphicx} 
\usepackage[left=2cm, right=2cm, bottom=2cm, top=2.5cm]{geometry}
%\usepackage[left=1in, right=1in, bottom=1in, top=1in]{geometry}

% \pagenumbering{arabic}

\usepackage{fancyhdr}
\pagestyle{fancy}
% \renewcommand{\headrulewidth}{0pt}
\renewcommand{\footrulewidth}{0pt}
% \lfoot{M. Bontrager}
% \rfoot{CV \thepage}
\cfoot{}

\lhead{Megan Bontrager---CV \thepage}
% \rhead{CV \thepage}
\chead{}
% \pagenumbering{gobble} 

\renewcommand\tabcolsep{0.2cm}

\usepackage[none]{hyphenat}

\usepackage{array}

\usepackage{enumitem}

\usepackage{hanging}

\newcommand\hangbibentry[1]{%
    \smallskip\par\hangpara{1em}{1}\bibentry{#1}\smallskip\par 
}

\newenvironment{absolutelynopagebreak}
  {\par\nobreak\vfil\penalty0\vfilneg
   \vtop\bgroup}
  {\par\xdef\tpd{\the\prevdepth}\egroup
   \prevdepth=\tpd}

%%% End preamble %%%


\begin{document}


\thispagestyle{empty}
%%% Personal info %%%

\noindent  \LARGE{\textbf{Megan Bontrager}} 
\smallskip
\normalsize

\noindent \begin{tabular}{@{} p{10cm} >{\raggedleft\arraybackslash}p{8.11cm}}
Postdoctoral Researcher \\
University of California, Davis & \\
Department of Evolution and Ecology & \\
{\href{mailto:mgbontrager@gmail.com}{mgbontrager@gmail.com}} & \\
{\href{https://meganbontrager.github.io}{meganbontrager.github.io}} & \\

\end{tabular}
\vspace{1em}

\noindent\hrulefill 

\bigskip
\bigskip



%%% Research interests %%%

\noindent \begin{tabular}{@{} p{3cm} >{\raggedright}p{14.2cm}}
% \noindent \begin{tabular}{@{} p{3cm} p{14.2cm}}
\Large{Research interests} & I study the determinants of species' geographic ranges and the drivers of local adaptation. I conduct large-scale, rigorous field and greenhouse experiments, quantitative syntheses of the the literature, and analyses using herbarium collections. \\
\end{tabular}
\bigskip
\bigskip



%%% Academic positions %%%

\noindent \begin{tabular}{@{} p{3cm} p{12cm} >{\raggedleft\arraybackslash}p{1.7cm}}
\Large{Academic}    & \textbf{Postdoctoral researcher} & 2018--\hspace*{0.8cm} \\
\Large{positions}   & University of California, Davis & \\
\end{tabular}

\noindent \begin{tabular}{@{} p{3cm} p{14.2cm}}
& Advisors: Jennifer Gremer, Julin Maloof, Johanna Schmitt, Sharon Strauss \\
%& \raggedright{Project themes:} \\
% \raggedright{\textit{1) How has the germination niche evolved across a clade of native mustards?}}  \\
% \raggedright{\textit{2) How does flowering phenology respond to interannual weather variation?}}  \\
%% rephrase
% \raggedright{\textit{3) How have vernalization requirements diverged along elevation gradients?}}  \\
\end{tabular}
\smallskip

\noindent \begin{tabular}{@{} p{3cm} p{12cm} >{\raggedleft\arraybackslash}p{1.7cm}}
& \textbf{Staff research associate} & 2011--2012 \\
& University of California, Santa Cruz & \\
& Supervisors: Ingrid Parker, Greg Gilbert &  \\
\end{tabular}

\bigskip
\bigskip

  
%%% Education %%%

\noindent \begin{tabular}{@{} p{3cm} p{12cm} >{\raggedleft\arraybackslash}p{1.7cm}}
\Large{Education}    & \textbf{University of British Columbia} & 2012--2018 \\
& \textbf{Ph.D.\ in Botany} & \\
\end{tabular}

\noindent \begin{tabular}{@{} p{3cm} p{13.7cm}}
& \raggedright{Advisor: Amy Angert} \\
\raggedright{Committee: Sally Aitken, Michael Whitlock, Jeannette Whitton} \\ 
\raggedright{Title: \textit{Pollination, genetic structure, and adaptation to climate across the geographic range of} Clarkia pulchella.} \\
\end{tabular}

\noindent \begin{tabular}{@{} p{3cm} p{12cm} >{\raggedleft\arraybackslash}p{1.7cm}}
& \textbf{University of California, Santa Cruz} & 2008--2011 \\
& \textbf{B.Sc.\ in Plant Sciences} & \\
& \textbf{B.Sc.\ in Molecular, Cell, and Developmental Biology} & \\
& Undergraduate research advisors: Kathleen Kay, Ingrid Parker  & \\
& & \\
& \textbf{Cabrillo Community College} &  2007--2008 \\
& Prerequisites for transfer to B.Sc. & \\
\end{tabular}
\bigskip
\bigskip


%%% Professional experience %%%




%%% Preprints %%%

\noindent\Large{Preprints} 
\normalsize
\bigskip

\def\arraystretch{1.4}
\noindent \begin{tabular}{@{} p{0.5cm} >{\raggedright\arraybackslash}p{16.7cm}}
14. & \textbf{M. Bontrager}, C. D. Muir, C. R. Mahony, D. E. Gamble$^{*}$, R. M. Germain, A. L. Hargreaves, E. J. Kleynhans, K. A. Thompson, and A. L. Angert. Climate warming weakens local adaptation. \textit{bioR$\chi$iv} 2020.11.01.364349. In review. \\
13. & \textbf{M. Bontrager}, J. A. Lee-Yaw, T. Usui, A. L. Hargreaves, D. Anstett, H. A. Branch, C. D. Muir, and A. L. Angert. Expansion dynamics and marginal climates drive adaptation across geographic ranges. \textit{bioR$\chi$iv} 2020.08.22.262915. In revision. \\
 \end{tabular}
 
 \smallskip
 
 \def\arraystretch{1.4}
\noindent \begin{tabular}{@{} p{0.5cm} >{\raggedright\arraybackslash}p{16.7cm}}
12. & \textbf{M. Bontrager} and A. L. Angert. Genetic differentiation is determined by geographic distance in \textit{Clarkia pulchella}. \textit{bioR$\chi$iv} 374454. In revision for resubmission. \\
\end{tabular}

\bigskip
\bigskip
\smallskip



%%% Publications %%%

\noindent\Large{Publications}  
\normalsize
\bigskip

\def\arraystretch{1.4}
\noindent \begin{tabular}{@{} p{0.5cm} >{\raggedright\arraybackslash}p{16.7cm}}
11. & A. L. Angert, \textbf{M. Bontrager}, and J. \AA gren (2020). What do we really know about adaptation at range edges? \textit{Annual Review of Ecology, Evolution, and Systematics} 51: 341-361. \\
10. & J. R. Gremer, A. Chiono, E. Suglia, \textbf{M. Bontrager}, L. Okafor, and J. Schmitt (2020). Variation in the seasonal germination niche across an elevational gradient: the role of germination cueing in current and future climates. \textit{American Journal of Botany}, 107(2): 350-363. \\
9. & A. L. Hargreaves, R. M. Germain, \textbf{M. Bontrager}, J. Persi, and A. L. Angert (2020). Local adaptation to biotic interactions: a meta-analysis across latitudes. \textit{The American Naturalist}, 195(3): 395-411. \\
8. & \textbf{M. Bontrager}, C. D. Muir, and A. L. Angert (2019). Geographic variation in reproductive assurance of \textit{Clarkia pulchella}. \textit{Oecologia}, 190(1): 59-67. \\
7. & \textbf{M. Bontrager} and A. L. Angert (2019). Gene flow improves fitness at a range edge under climate change. \textit{Evolution Letters}, 3(1): 55-68. \\
6. & D. E. Gamble$^{*}$, \textbf{M. Bontrager}, and A. L. Angert (2016). Floral trait variation and links to climate in the mixed-mating annual \textit{Clarkia pulchella}. \textit{Botany}, 96(7): 425-435. \\
5. & \textbf{M. Bontrager} and A. L. Angert (2016). Effects of range-wide variation in climate and isolation on floral traits and reproductive output of \textit{Clarkia pulchella}. \textit{American Journal of Botany}, 103(1): 10-21.  \\
4. & J. A. Lee-Yaw, H. M. Kharouba, \textbf{M. Bontrager}, C. Mahony, A. M. Cserg{\H{o}}, A. M. Noreen, Q. Li, R. Schuster, and A. L. Angert (2016). A synthesis of transplant experiments and ecological niche models suggests that range limits are often niche limits. \textit{Ecology Letters}, 19(6): 710-722. \\
3. & I. M. Parker, M. Saunders, \textbf{M. Bontrager}, A. P. Weitz, R. Hendricks, R. Magarey, K. Suiter, and G. S. Gilbert (2015). Phylogenetic structure and host abundance drive disease pressure in communities. \textit{Nature}, 520(7548): 542-544. \\
2. & \textbf{M. Bontrager}, K. Webster, M. Elvin, and I. M. Parker (2014). The effects of habitat and competitive/facilitative interactions on reintroduction success of the endangered wetland herb, \textit{Arenaria paludicola}. \textit{Plant Ecology}, 215(4): 467-478. \\
1. & J. M. Yost, \textbf{M. Bontrager}, S. W. McCabe, D. Burton, M. G. Simpson, K. M. Kay, and M. Ritter (2013). Phylogenetic relationships and evolution in \textit{Dudleya} (Crassulaceae). \textit{Systematic Botany}, 38(4): 1096-1104. \\
\end{tabular}

\smallskip
\noindent $^{*}$ Undergraduate trainee for whom I was the primary mentor

\bigskip
\bigskip
\smallskip



%%% Non-refereed contributions %%%

\noindent\Large{Non-refereed contributions}  
\normalsize
\bigskip

\def\arraystretch{1.4}
\noindent \begin{tabular}{@{} p{0.5cm} >{\raggedright\arraybackslash}p{16.7cm}}
3. & K. R. Acierto, R. S. Hendricks, \textbf{M. Bontrager}, and I. M. Parker (12 December 2012). Transplant success for the endangered herb \textit{Arenaria paludicola} at Golden Gate National Recreation Area: effects of site, propagation type, and competition. Technical report to the U.S. Fish and Wildlife Service and the California Department of Fish and Game. \\
\end{tabular}

\def\arraystretch{1.4}
\noindent \begin{tabular}{@{} p{0.5cm} >{\raggedright\arraybackslash}p{16.7cm}}
2. & I. M. Parker and \textbf{M. Bontrager} (29 February 2012). Propagation and establishment of new populations of marsh sandwort (\textit{Arenaria paludicola}) in Santa Cruz County. Technical report to the U.S. Fish and Wildlife Service and the California Department of Fish and Game. \\
1. & \textbf{M. Bontrager} and I. M. Parker (26 September 2011). Effects of serpentine soil on plant community composition in natural populations and seedling growth in a bioassay. Technical report to Midpeninsula Regional Open Space District. \\
\end{tabular}

\bigskip
\bigskip
\smallskip



%%% Invited seminars %%%

\noindent\Large{Invited seminars} 

\normalsize
\bigskip

\def\arraystretch{1.3}
\noindent \begin{tabular}{@{} >{\raggedright\arraybackslash}p{17.2cm}}
Duke University, PopBio Seminar Series, 15 October 2020.\\
University of Utah, Frontiers in Plant Biology Symposium, 19 February 2020.\\
Hamilton Symposium at Evolution, Providence, Rhode Island, 28 June 2019. \href{https://www.youtube.com/watch?v=UeK_zYEfVyA}{Video link.} \\
University of California, Davis, Population Biology Seminar Series, 26 February 2019. \\
Maladaptation Symposium at the American Society of Naturalists Asilomar Meeting, 6 January 2018.
\end{tabular}
\bigskip
\bigskip
\smallskip



%%% Selected presentations %%%

\noindent\Large{Selected presentations}  
\normalsize
\bigskip

\def\arraystretch{1.4}
\noindent \begin{tabular}{@{} >{\raggedright\arraybackslash}p{17.2cm}}
\hangindent=5mm\textbf{M. Bontrager}, J. Maloof, J. R. Gremer, and S. Y. Strauss (4 January 2020). Climatic drivers of the flowering niche in the \textit{Streptanthus} clade. Poster presentation at the American Society of Naturalists meeting. Asilomar, California. \\
% \textbf{M. Bontrager}, C.R. Mahony, D.E. Gamble, R.M. Germain, A.L. Hargreaves, E.J. Kleynhans, C.S. Leven, K.A. Thompson, and A.L. Angert (6 January 2018). Climate anomalies drive local maladaptation. Presentation at the American Society of Naturalists meeting. Asilomar, California. \\
\hangindent=5mm\textbf{M. Bontrager} and A. L. Angert (4 April 2018). Effects of gene flow on performance  at the northern range margin of Clarkia pulchella. Presentation at Evo-Wibo. Port Townsend, Washington. \\
\hangindent=5mm\textbf{M. Bontrager} and A. L. Angert (24 June 2017). Effects of gene flow on the performance of Clarkia pulchella at the species’ northern range margin. Presentation at Evolution. Portland, Oregon. \href{https://www.youtube.com/watch?v=HqVgQzIJLyA}{Video link.} \\
\hangindent=5mm\textbf{M. Bontrager} and A. L. Angert (9 May 2017). Effects of gene flow on the performance of Clarkia pulchella at the species’ northern range margin. Presentation at the Annual Meeting of the Canadian Society for Ecology and Evolution. Victoria, British Columbia. \\
\end{tabular}

\def\arraystretch{1.2}
\noindent \begin{tabular}{@{} >{\raggedright\arraybackslash}p{17.2cm}}
\hangindent=5mm\textbf{M. Bontrager} and A. L. Angert (5 November 2016). Effects of gene flow on the performance of Clarkia pulchella at the species’ northern range margin. Presentation at Ecology and Evolution Retreat. Brackendale, British Columbia. \\
\hangindent=5mm\textbf{M. Bontrager} and A. L. Angert (16 April 2016). Effects of gene flow on the performance of Clarkia pulchella at the species’ northern range margin. Poster presentation at Evo-Wibo. Port Townsend, Washington. \\
\hangindent=5mm\textbf{M. Bontrager} and A. L. Angert (22 May 2015). Effects of range-wide variation in climate and isolation on floral traits and reproductive output of Clarkia pulchella. Presentation at the Annual Meeting of the Canadian Society for Ecology and Evolution. Saskatoon, Saskatchewan. \\
\hangindent=5mm\textbf{M. Bontrager}, K. Webster, M. Elvin, and I. M. Parker (12 January 2012). Factors influencing growth and survival of a critically endangered plant, Arenaria paludicola. Presentation at the California Native Plant Society 2012 Conservation Conference. San Diego, California. \\
\hangindent=5mmJ. Yost, \textbf{M. Bontrager} (co-presented), S. McCabe, K. M. Kay, and M. Ritter (11 July 2011). A classification of California’s diploid Dudleya species based on molecular phylogenetic data. Poster presentation at Botany 2011 Conference. St. Louis, Missouri. \\
\end{tabular}
\bigskip
\bigskip


%%% Fellowships and awards %%%

\pagebreak

\noindent\Large{Fellowships and awards}
\normalsize
\bigskip

\noindent \begin{tabular}{@{} >{\raggedright\arraybackslash}p{16cm} >{\raggedleft\arraybackslash}p{1.2cm}}
Society for the Study of Evolution Hamilton Finalist (500 USD) & 2019 \\ 
Grand Challenges Postdoctoral Fellowship, University of Minnesota (declined; 107,000 USD) & 2018 \\
UBC Biology teaching award (500 CAD) & 2018 \\
Student talk award, Evo-Wibo, Port Townsend, Washington & 2018 \\
Best research presentation, Brackendale Ecology and Evolution Retreat & 2016 \\
Li Tze Fong Memorial Fellowship (25,000 CAD) & 2016 \\
Botanical Society of America Genetics Section Grad Research Award (500 USD) & 2016 \\
Botanical Society of America Graduate Student Research Award (500 USD) & 2016 \\
Washington Native Plant Society Research Grant (1,200 USD) & 2016 \\
Vladimir J. Krajina Prize in Plant Ecology (2,000 CAD) & 2013 \\
UBC Four Year Doctoral Fellowship (102,400 CAD) & 2012 \\
\end{tabular}
\bigskip
\bigskip


%%% Mentoring and teaching %%%

\noindent\Large{Mentoring and teaching}
\normalsize
\bigskip

\def\arraystretch{1.1}
\noindent \begin{tabular}{@{} >{\raggedright\arraybackslash}p{15.5cm} >{\raggedright\arraybackslash}p{1.7cm}}

 \textbf{Teaching experience}  & \\
  Lead teaching assistant, Biostatistics (UBC, 2 terms) & 2017--2018 \\
  \hspace{5mm}\textit{Coordinated all TAs and prepared written guides for running labs} &  \\
  \hspace{5mm}\textit{Assisted with writing exams and provided feedback on course materials in development} &  \\
  \hspace{5mm}\textit{Received UBC Biology Teaching Award for outstanding work in this role} &  \\
 Teaching assistant, Plant Ecology (UBC) & 2017 \\
  \hspace{5mm}\textit{Developed lab activities in data collection and analysis} & \\
  \hspace{5mm}\textit{Led labs in the field, greenhouse, and on the computer} &  \\  
  \hspace{5mm}\textit{Facilitated discussions of primary literature} &   \\
 Teaching assistant and guest lecturer, Phytogeography (UBC) & 2016 \\ 
  \hspace{5mm}\textit{Provided suggestions for revisions on written work} &   \\
  \hspace{5mm}\textit{Facilitated discussions of primary literature} &   \\
\end{tabular}
\smallskip

\def\arraystretch{1.1}
\noindent \begin{tabular}{@{} >{\raggedright\arraybackslash}p{15.5cm} >{\raggedright\arraybackslash}p{1.7cm}}
 \textbf{Mentoring experience}  & \\
 Supervisor and mentor to post-baccalaureate lab technicians (UC Davis) & 2018--2020 \\
 Advisor to undergraduate students (UC Davis) & 2018-- \\
  \hspace{5mm}\textit{5 students presented work at the Undergraduate Research Conference} &  \\
  \hspace{5mm}\textit{1 ongoing student project} &  \\
  \hspace{5mm}\textit{8 additional students trained and mentored} &  \\
   % Project students with posters: Annie Adachi, Siwon Chung, Adrianna Ng, William Smith, Lila Simpson
 % Ongoing project students: Adrianna Ng
 % Other students: Evan Jordan, Lara Hsia, Kees Hall, Ian Clark, Eda Ceviker, Louisa Liu, Natascha Paxton, Maya Martinez
 Co-advisor of undergraduate honours thesis students (UBC, 2 students) & 2016--2017 \\
 % Devin Gamble (currently PhD student at UCSB), Carolina Sato (currently MSc student at UBC)
 Supervisor of undergraduate research assistants (UBC, 4 students) & 2014--2017 \\
 % Erin Fitz, Catriona Leven, Joyce Chen, Devin Gamble
 Supervisor of undergraduate research assistants (UC Santa Cruz, 3 students) & 2011--2012 \\
 \end{tabular}
\smallskip

\def\arraystretch{1.1}
\noindent \begin{tabular}{@{} >{\raggedright\arraybackslash}p{15.5cm} >{\raggedright\arraybackslash}p{1.7cm}}
 \textbf{Workshops given}  & \\
 Leader and developer, Data management workshop (for colleagues at UC Davis) & 2020 \\
 Leader and developer, Intro to R workshop (for undergraduate researchers at UC Davis) & 2018 \\
  \end{tabular}
\smallskip

\def\arraystretch{1.1}
\noindent \begin{tabular}{@{} >{\raggedright\arraybackslash}p{15.5cm} >{\raggedright\arraybackslash}p{1.7cm}}
 \textbf{Pedagogical training}  & \\
 Participant, Center for Educational Effectiveness Accelerate Program, UC Davis  & 2020 \\
 Participant, Education Research and Evidence-based Teaching, UC Davis  & 2020 \\

\end{tabular}
\bigskip
\bigskip


\pagebreak
%%% Service and outreach %%%

\noindent\Large{Service, outreach, and professional development}  
\normalsize
\bigskip

\def\arraystretch{1.1}
\noindent \begin{tabular}{@{} >{\raggedright\arraybackslash}p{15.5cm} >{\raggedright\arraybackslash}p{1.7cm}}
 Mentor, Evolution and Ecology Graduate School Preview, UC Davis & 2020 \\
 Participant, Anti-Racism reading group & 2020 \\
 Administrative member, Women in Life Sciences at UC Davis & 2019--2020 \\
 Mentor, Evolution and Ecology Graduate Admissions Pathways, UC Davis & 2019 \\
 Grad representative, Biodiversity Research Centre postdoc search committee, UBC & 2018 \\
 Co-organizer, Biodiversity Centre Women in STEM Workshop, UBC & 2017 \\
 Coordinator of Florum, a weekly meeting of plant ecologists, UBC & 2013--2016 \\
 Curriculum developer, Modules in Ecology and Evolution Development, UBC & 2013--2015 \\
  \hspace{5mm}\textit{Developed educational activity about pollination, presented it in a primary school} \\  \hspace{5mm}\textit{classroom, and added it to a library of activities for future use.} \\
 Visiting scientist in primary school classrooms, Let's Talk Science, UBC & 2012--2014 \\
   \hspace{5mm}\textit{Led educational activities for students over the course of several visits in two classrooms.} \\
 Science fair mentor, Let's Talk Science, UBC & 2012--2013 \\
   \hspace{5mm}\textit{Mentored two high school students from project design through to their presentation.} \\
 Volunteer, Beaty Biodiversity Museum Nature Club, UBC & 2012--2013 \\ 
   \hspace{5mm}\textit{Facilitated educational activities for kids and their families several weekends per term.} \\
\end{tabular}
\bigskip


\noindent\Large{Selected field experience}  
\normalsize
\bigskip

\def\arraystretch{1.1}
\noindent \begin{tabular}{@{} >{\raggedright\arraybackslash}p{15.5cm} >{\raggedright\arraybackslash}p{1.7cm}}
Demographic surveys of \textit{Streptanthus tortuosus} & 2019-2020 \\
Transplant installation and monitoring (\textit{Clarkia pulchella}) & 2015-2016 \\
\hspace{5mm}\textit{Obtained permits, coordinated logistics for planting and relocating 16,800 seeds} & \\
Water addition and pollinator exclusions to experimental plots (\textit{Clarkia pulchella}) & 2016 \\
\hspace{5mm}\textit{Obtained permits, coordinated logistics for 8 sites in 3 states/provinces} & \\
Demographic surveys of \textit{Mimulus cardinalis} & 2013-2016 \\
\hspace{5mm}\textit{Multi-week trips relocating and surveying populations in rugged riparian terrain} & \\
Surveys of plant community composition and abundance & 2010-2011 \\
\hspace{5mm}\textit{Identified every plant found in chaparral, redwood forest, and grassland habitats} &\\
\hspace{5mm}\textit{Developed quantitative methods, trained and led field crews} & \\
Transplant installation and monitoring reintroduction of \textit{Arenaria paludicola} & 2010-2011 \\
\hspace{5mm}\textit{Characterized biotic and abiotic site conditions in wetland habitat} & \\
\hspace{5mm}\textit{Prepared reports on this locally extirpated species for government agencies} & \\
Evaluate invasive species removal methods for reforestation & 2009-2010 \\
\hspace{5mm}\textit{Characterize density and size classes of Cytisus scoparius} & \\
\end{tabular}
\bigskip


%%% Professional memberships %%%

\noindent\Large{Professional engagement}  
\normalsize
\bigskip

\noindent{\textbf{Reviews since 2019:} American Journal of Botany (1), Ecology Letters (1), Evolution (3), Evolution Letters (1), Global Change Biology (2), Global Ecology and Biogeography (1), Journal of Ecology (2), Journal of Systematics and Evolution (1), New Phytologist (2), PeerJ (2), Trends in Ecology and Evolution (1).}

\bigskip

\noindent{\textbf{Member:} American Society of Naturalists, Botanical Society of America, Canadian Society for Ecology and Evolution, Society for the Study of Evolution, Washington Native Plant Society}

\bigskip
\bigskip







%%% Selected graduate courses %%%

% \noindent \begin{tabular}{@{} p{3cm} p{11.21cm} >{\raggedleft\arraybackslash}p{1.7cm}}
% \Large{Selected} & Bioinformatics for evolutionary biology (Greg Owens, Kay Hodgins) & 2016 \\
% \Large{graduate} & Data management for biologists (Andrew MacDonald) & 2016 \\
% \Large{courses}  & Population ecology (Amy Angert) & 2015 \\
%  & Species distribution modelling (Tom Edwards) & 2014 \\
%  & Data analysis in R (Thor Veen) & 2013 \\
%  & Population genetics (Michael Whitlock) & 2012 \\
% \end{tabular}
% \bigskip
% \bigskip




%%% Relevant skills %%%

% \normalsize
% \noindent \begin{tabular}{@{} p{3cm} p{13.11cm}}
% \Large{Relevant} & \textbf{Lab:} DNA extraction, RADseq library prep. \\
% \Large{skills} & \textbf{Data management and statistics:} advanced use of R (including mixed-effect statistical models, species distribution models, and GIS-type uses), Github, Latex, proficiency in Unix command line. \\
%  & \textbf{Bioinformatics:} sequence alignment and SNP identification with the Stacks pipeline. \\
%  & \textbf{Field:} installation and monitoring of large field experiments, demographic surveys, plant community composition surveys. \\
%  & \textbf{Greenhouse:} plant care, trait measurements, and controlled pollinations. \\
% \end{tabular}
% \bigskip
% \bigskip

% \noindent \begin{tabular}{@{} p{3cm} p{10.91cm} >{\raggedleft\arraybackslash}p{1.7cm}}
% & \textbf{Undergraduate research assistant} & 2010--2011 \\
% & \textbf{University of California, Santa Cruz} &  \\
% & Supervisors: Ingrid Parker, Kathleen Kay, and Jenn Yost & \\
% % Developed and performed molecular protocols and greenhouse experiments.
% % Assisted with field experiments. & \\
% \end{tabular}
% 
% \bigskip

%\noindent \begin{tabular}{@{} p{3cm} p{12cm} >{\raggedleft\arraybackslash}p{1.7cm}}
%& \textbf{Staff research associate} & 2011--2012 \\
%& University of California, Santa Cruz & \\
%& Supervisors: Ingrid Parker, Greg Gilbert &  \\
% \hspace{10mm}Trained and led crews in the field, greenhouse, and molecular biology lab.  & \\
% \hspace{10mm}Designed greenhouse experiments and field monitoring protocols. & \\
%\end{tabular}

\end{document}
